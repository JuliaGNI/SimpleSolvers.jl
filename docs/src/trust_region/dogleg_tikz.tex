\documentclass[tikz, border=10pt]{standalone}
\usetikzlibrary{arrows.meta, intersections, calc}

\begin{document}
\begin{tikzpicture}[scale=1.5]
    % Coordinates
    \coordinate (O) at (0,0);
    \coordinate (PU) at (0,-1.5);    % Cauchy point (Steepest Descent)
    \coordinate (PB) at (3.5,-0.5); % Newton Point

    % Draw Trust Region (Circle)
    \def\radius{2.5}
    \draw[thick, dashed, color=gray!60] (O) circle (\radius);
    \node[color=gray!80, anchor=north west] at (1.8, 1.8) {Trust Region ($\Delta_k$)};

    % Draw Axes (optional, for context)
    \draw[->, color=gray!30] (-1,0) -- (4,0);
    \draw[->, color=gray!30] (0,1) -- (0,-2.5);

    % Draw the Dogleg Path
    % Segment 1: Origin to Cauchy Point
    \draw[thick, blue] (O) -- (PU);
    % Segment 2: Cauchy Point to Newton Point
    \draw[thick, blue, name path=dogleg_seg] (PU) -- (PB);

    % Find intersection of the path with the circle
    \path[name path=circle] (O) circle (\radius);
    \path [name intersections={of=dogleg_seg and circle, by={pk}}];

    % Mark the points
    \filldraw[black] (O) circle (1.5pt) node[anchor=south east] {$x_k$};
    \filldraw[blue] (PU) circle (1.5pt) node[anchor=east, xshift=-2pt] {$p^U$};
    \filldraw[red] (PB) circle (1.5pt) node[anchor=west] {$p^B$ (Newton point)};
    \filldraw[black] (pk) circle (1.8pt) node[anchor=south west, yshift=-9pt] {$p_k$ (Dogleg step)};

    % Labels for the path
    \draw[<-, >=Stealth, shorten <=3pt] (1.2, -1.1) -- (1.8, -1.8)
        node[anchor=north] {Dogleg Path};

    % Highlighting the gradient direction
    \draw[->, thick, gray] (O) -- (0,-0.7) node[midway, left] {$-g_k$};

\end{tikzpicture}
\end{document}
